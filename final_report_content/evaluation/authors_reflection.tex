% Talk about project aims
This section provides an evaluation of the project as a whole from the perspective of the developer. This takes into account the project aims, requirements, and motivations, and uses the questions set out below to assist with this evaluation.


\subsubsection{What is the (technical) contribution of this project?}
This project aimed to develop a system which could model FPGA-based smart network switches. This would help users understand the potential of these currently theoretical devices, and hopefully assist in their development.
The system developed by the project provides an easy to use, lightweight, configurable way to model FPGA-based smart network switches, providing options to make the flow of data in the model more similar to that in real networks.
The strong focus on maintaining good practices during development, as well as making the project open-source, should significantly improve the maintainability of the software, and should enable it to be used in future work where appropriate, such as that described in section \ref{future_work}.

\subsubsection{Why should this contribution be considered relevant and important for the subject of your degree?}
This project demonstrates a clear interconnection between the two departments which make up \textit{Computer Systems Engineering}.
While well used in Computer Science, the development of FPGAs is a significant field of Digital Electronic Engineering, as has been shown through the Warwick Engineering modules \textit{ES3B2} \cite{es3b2} and \textit{ES3F1} \cite{es3f1}.
In addition, computer networks are a significant field of Computer Science.
These two fields are both of great interest to the developer, which was a motivation for the initial project title and ideas.
This project can demonstrate the value of the \textit{Computer Systems Engineering} degree, since a project involving the areas examined by this project requires background knowledge in the areas, and that can only be acquired from this course.
It is hoped that this project will contribute to the development of FPGA-based smart network switches, which would fall well within the objectives of the developer's degree.

\subsubsection{How can others make use of the work in this project?}
Since the system developed for this project is open-source and available publicly on \textit{GitHub}, anyone can make use of the code. This could either be by downloading and using the code as is, for which detailed instructions are clearly displayed on the webpage, or by using it as a basis for a related project.

The research conducted in this project identifies some applications of FPGA-based smart network switches, as well as a number of ways to help develop these systems. Should further research occur to develop FPGA-based smart network switches, this project could be used to provide design assistance, as well as advice on the potential pitfalls.

\subsubsection{Why should this project be considered an achievement?}
This project investigates an area which is currently entirely theoretical. While this is a common situation in academia, the constraints placed on the project, in particular the limited time frame, made finding relevant materials challenging. Regardless, relevant research has been conducted, and a modelling system has been produced which meets all of the functional and non-functional requirements defined for it. The project's agile software development methodology allowed for flexibility and change where appropriate, preventing the project from becoming stuck on a particular path if it became unfeasible. Finally, the project should be able to contribute positively to further research around the subject of the project.

\subsubsection{What are the limitations of this project?}
The primary limitation of this project is that the implemented aspect is a model, rather than a working system. Therefore, results from the model cannot be totally relied on, since they are only simulations and will never be exactly accurate compared to real data. In addition, unlike a raw instance of \textit{Mininet}, the model does not allow users to specify completely customisable topologies. While many topologies will be a subset of a tree, the current implementation cannot produce topologies containing, for example, cycles.
