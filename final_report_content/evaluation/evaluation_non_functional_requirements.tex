% To what extent did you achieve the non functional requirement
% Make reference to it
% Draw table with SUCCESS or FAILURE, and evaluation of how the requirement has been met

This section evaluates the success of the non-functional requirements laid out in section \ref{non_functional_requirements}. Table \ref{evaluation_non_functional_requirements_table} displays each requirement along with a statement of whether or not it has been successfully achieved.

\begin{center}
  \begin{longtable}{|l|p{9cm}|c|}
    \caption{Success of Non-Functional Requirements} \\ \hline
    \label{evaluation_non_functional_requirements_table}
    \textbf{NF\#} & \textbf{Requirement} & \textbf{Success / Failure} \\ \thickhline
    \textbf{NF1} & The system \textbf{must} be modular, to allow extensibility. & \textcolor{green}{SUCCESS} \\ \hline
    \textbf{NF2} & The system \textbf{should} be scalable. & \textcolor{green}{SUCCESS} \\ \hline
    \textbf{NF3} & All third-party libraries used by the system \textbf{should} be open-source. & \textcolor{green}{SUCCESS} \\ \hline
    \textbf{NF4} & The system \textbf{should} be efficient, able to run all tests on relatively large virtual networks in small amounts of time. & \textcolor{green}{SUCCESS} \\ \hline
    \textbf{NF5} & All code for the system \textbf{should} be well-documented and maintainable. & \textcolor{green}{SUCCESS} \\ \hline
  \end{longtable}
\end{center}
