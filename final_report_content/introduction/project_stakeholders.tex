% Manufacturers of devices
  % FPGAs
    % Xilinx
    % Altera
  % Switches
    % HP
    % Cisco
    % Netgear
    % D-Link
% Users
  % Hospitals
  % Schools
  % Offices
  % Military
  % within datacentres - data going around centre
  % IoT
% Academia
  % Expanding research area?
  % NetFPGA community?
  % Mininet commuity?
% People involved in project
  % Me
  % Suhaib

\subsection{Primary Stakeholders}
\label{primary_stakeholders}
This project had two primary stakeholders who contributed to the development of the project. These are the developer, Benji Levine, and the project supervisor, Suhaib Fahmy. Through regular meetings and collaboration, the primary stakeholders were able to produce the fundamental ideas behind the project, and develop them across the project's duration. As a result, both have a clear stake in the project.

\subsection{Secondary Stakeholders}
\label{secondary_stakeholders}
The secondary stakeholders are considered to be those who could benefit from the success of the project. Since this project is primarily research based, many of the secondary stakeholders would only benefit from further development of this project.

\subsubsection{Device Manufacturers}
Should FPGA-based smart network switches become successful, manufacturers of FPGAs such as \textit{Xilinx} \cite{xilinx} and \textit{Intel} \cite{intel_fpga}, as well as manufacturers of network switches such as \textit{Cisco} \cite{cisco} and \textit{Netgear} \cite{netgear}, would all benefit, both financially and in terms of additional publicity.

\subsubsection{Users}
These devices are aimed many different types of user, however they could perform particularly well for either users relying on cloud computing platforms for data processing, or IoT. IoT tends to involve small payloads of data, however potentially large quantities of these payloads. FPGA-based smart network switches are suited very well for data formatted in this way.
An example application could be tracking medical equipment in a hospital. All equipment, from small items such as scalpels, to large items such as MRI machines, could be fitted with a small, low-power IoT tracking device, and then tracking devices could be fitted throughout the existing hospital infrastructure, such as in the lighting equipment. This method would result in all items being locatable instantly, and by adding FPGA-based smart network switches into this architecture, this processing can happen much faster, and within the existing infrastructure.
With the exception of the FPGA-based smart network switches, this equipment is all available, such as \textit{Electric Imp's} \textit{Asset Tracking through Ambient Lighting} \cite{electric_imp} \cite{electric_imp_asset_tracking}.

This use case is applicable to many different sectors, such as education, military, healthcare, and corporate environments, and could also be expanded out to private individuals in future as hardware costs reduce.

\subsubsection{Academia}
FPGAs and networking are both very active areas of research, and it is hoped that this project will contribute positively. This project involved both the \textit{Mininet} project \cite{mininet} and the \textit{NetFPGA} project \cite{NetFPGA}. Both of these projects are under active development by academic communities, and would be heavily involved in further implementation of FPGA-based smart network switches. 
