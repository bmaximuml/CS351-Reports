% This is what we're aiming to do with the project.
% In our case, it's create this python script.
% It's not, it's the FPGA-based smart switches. The python script is simply the implemented aspect.

% So what we want to be doing is talking about the research.
%   Use words like investigate a lot.
%   Look at the aims of the other projects.

% It's also very important to talk about the reasoning behind all of the aims

% Create an application to model networks containing FPGA-based smart switches
% Allow users to test whether FPGA smart switches would be suitable in their network

% Something to do with implementing on NetFPGA?
% Something to do with SDN, control plane


% Coles:
%   Connect platforms
%   Connect people
%   use simple techniques
%   provide recommendations

%%% STRUCTURE %%%%

% Coles
  % Different subsections for each aim
  % 3 - 6 aims

% Dutton
  % no subsections
  % 1 overarching aim, 1-2 sub aims

% Thompson
  % Talk about how the aims changed after the project started
  % i.e. initially the aims were to build the hardware, but that changed to be more research focus


The main aim of this project is to research the complexities of creating an FPGA-based smart network switch. This is a notable change from the initial aims of the project, which were more focused on implementation rather than research. However, due to the agile nature of the project, discussed further in section \ref{project_management}, it was possible to make this change.
The initial implementation-focused aims of the project were found to limit the project from exploring certain areas of research which proved both relevant and interesting. In addition, re-evaluating the feasibilty of the initial project aims after meeting certain external delays, even while using the risk mitigation strategies laid out in section \ref{risk_management}, indicated that that these updated project aims would be less susceptible to similar delays.

\subsection{Research FPGA-Based Smart Network Switch}
This is the primary aim of the project. These devices are currently entirely theoretical, and as a result more research than is feasible in this project will be required before these devices are close to being commercially available. As a result, the research into these devices, along with the complications of their design and implementation is being prioritised over the actual implementation of one such device. This will allow for the project to investigate the theory of FPGA-based smart network switches in more depth.

\subsection{Model FPGA-Based Smart Network Switches}
A further aim of the project is to design and implement a model of the FPGA-based smart network switches in order to demonstrate their potential advantages over a standard connection to a server in a data centre. This should allow users to test whether FPGA-based smart network switches would be suitable in their network, and to design their network architecture accordingly if they are.
