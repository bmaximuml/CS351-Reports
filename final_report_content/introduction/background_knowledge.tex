Prior to the commencement of this project, some existing knowledge was used to assist in choosing the direction for the project, and finding different research areas. These are detailed below.

\subsection{Networking Concepts}
\label{networking_concepts_introduction}
The Warwick Computer Science module \textit{Operating Systems and Computer Networks} \cite{cs241} provided a background for many basic networking concepts, including the \textit{OSI network model} and the \textit{TCP/IP network model}, both of which are further discussed in section \ref{network_models}. This module aimed to provide an understanding of the purpose of each layer of the \textit{OSI network model} (shown in table \ref{osi_model}), and where this information has been of relevance to this project, it has been used as a basis for the research shown in section \ref{research}.

Working with networking hardware, including network switches, as part of \textit{Warwick Student Cinema} \cite{wsc} provided practical experience in using this equipment. This proved useful throughout the project, as it allowed for an additional viewpoint to be taken towards the project, i.e. that of someone who manages network infrastructure.
% Thinking about how the hardware side of the project would integrate with existing infrastructure is an important consideration, as the marketability of the project will be limited if it does.
Ensuring the hardware side of the project integrates sufficiently with existing infrastructure is important, since the marketability of the project will be limited if it fails to do so.

This experience, enhanced by a recent visit to a datacentre, provided a basic understanding of the different physical media used in network cables. The primary mediums are UTP made of copper, or fibre optics, often referred to simply as ``fibre''. This has been researched further and is discussed in more detail in section \ref{physical_media_research}.

% Talk about OSI model
  % Reference OSN
  %
% Talk about TCP/IP model
  % Reference OSN

% Talk about experience with network switches

% Copper vs Fibre?



\subsection{FPGAs}
\label{fpgas}

The Warwick Engineering modules \textit{Digital Systems Design} \cite{es3b2} and \textit{High Performance Embedded Systems Design} \cite{es3f1} provided the core concepts of FPGAs. This included the FPGA's basic structure of LUTs, block memories, DSP blocks, and IO, how FPGAs compared to ASICs and CPUs in terms of speed, flexibility, and price, different applications of FPGAs, and more. This background was crucial for the development of the project, and motivated many of the initial targets and aims it.


% ES3B2
% Basic structure, LUTs, block memories, DSPs, IO
% Flexibility: ASIC < FPGA < CPU
% Speed: ASIC > FPGA > CPU
