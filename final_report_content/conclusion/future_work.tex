% Previously titled Further Development
% Now titled Future Work



% Link to motivations
% Website
% Social media
% AI/ml
% Hardware
% other FPGA areas
% partial reconfiguration
%
% any requirements that haven't yet been met
%
% Usage guide
% More deployment methods?
%
% scalability

The research and system produced by this produced can be taken further in a variety of different ways. Some of these relate directly to the implemented system, while others are more abstract ways in which this work could be further developed.

\subsection{Implementing an FPGA-based Smart Network Switch}
The most obvious avenue for future work is to implement an FPGA-based smart network switch. A prototype for this type of system would require an FPGA integrated with networking hardware, such as one of the \textit{NetFPGA} platforms described in section \ref{netfpga}. It would also require a packet switching SDN language, such as \textit{P4}, as described in section \ref{p4}, and this language would need to be used to analyse packets up to the Application layer of the \textit{TCP/IP network model}, as described in section \ref{network_models}, and determine whether these packets should be sent to the FPGA for processing, or simply routed through the network as normal. Furthermore, code would need to be written to perform the packet processing on the FPGA.

\subsection{GUI for Implemented System}
A GUI could make the implemented system more easy to use, particularly for those without easy access to the environment required to run the system. This could involve a drag-and-drop interface to design a network topology, which would then be converted to the \textit{Python} format required for \textit{Mininet}'s API. This could be deployed to a website, allowing people to save and share the network diagrams they create and the performance tests they run. This would naturally add the requirement of hosting the website.

\subsection{API for Implemented System}
An API for the implemented system would make it easier to integrate the application into further developed applications, such as a GUI, or an application to add futher functionality. This could be generated in multiple programming languages, and would make it easier for those interested in adding functionality to this system to do so.

\subsection{Realistic Randomisation for Implemented System}
In the real world, external factors have an impact on the passage of data through a network. This can have an unpredictable effect on the bandwidth, average latency, and average packet loss of a network. While the currently implemented system does allow for these attributes to be customised, they can currently only be set to a constant value or a value taken from a Poisson distribution. Introducing a greater element of randomness as option for users into the system will bring it closer to a real network, and therefore make it more practically useful.
