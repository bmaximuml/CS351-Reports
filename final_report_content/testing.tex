% Travis
% Unit tests
% Integration tests
% system testing
% user testing
%   Show to isabelle, ask her what she thinks
%
% can be installed from source on clean install
% deploy to docker hub, testing there?
% deploy to pip, testing there?
%

Comprehensive tests were necessary in order to ensure that the implemented system functioned as intended.
The testing for this system was separated into unit, integration, system, environment, and user testing, as discussed further in the sections below.

As mentioned in section \ref{testing_structure}, \textit{Travis CI} \cite{travis_ci} is a web-based CI platform designed for running automated tests.
It was used for many of the tests in this project due to its clear interface, and easy integration with \textit{GitHub} repositories.
Whenever code was uploaded to \textit{GitHub}, \textit{Travis CI} would automatically run all configured tests on the code, and would notify the developer by email if any tests failed.
The testing environment used by \textit{Travis CI} can be easily configured using a single \textit{YAML} \cite{yaml} file located in the \textit{GitHub} repository for the project.



\section{Unit Testing}
\label{unit_testing}
Unit testing was conducted to ensure that each individual component of the system functioned as intended.
Tests were written for function in the code using a variety of different inputs, and the output of these tests was compared with the expected output.
The \textit{Python} library \textit{unittest} \cite{python_unittests} was used to write these tests, as it is designed for this purpose, and integrates easily with both the existing system and \textit{Travis CI}.
The results of the unit tests can be seen in a simplifed format in table \ref{unit_test_results}.

\begin{table}[t]
  \caption{Results of Unit Tests}
  \begin{center}
    \begin{tabularx}{\textwidth}{|Y|c|} \hline
      % \textbf{Unit Test} & \makecell{\textbf{Linked \\ Requirement(s)}} & \textbf{Success / Failure} \\ \thickhline
      \textbf{Unit Test} & \textbf{Success / Failure} \\ \thickhline
      Test \textit{halve\_delay} function & \textcolor{green}{SUCCESS} \\ \hline
      Test \textit{get\_poisson\_delay} function & \textcolor{green}{SUCCESS} \\ \hline
      Test \textit{test\_cloud\_fpga} function & \textcolor{green}{SUCCESS} \\ \hline
      Test \textit{TreeTopoGeneric} class & \textcolor{green}{SUCCESS} \\ \hline
      % 0 & 0 & 0 & 0 (A) \\ \hline
    \end{tabularx}
  \end{center}
  \label{unit_test_results}
\end{table}

\newpage

\section{Integration Testing}
\label{integration_testing}
Integration testing was conducted to ensure the individual components of the system interacted with each other as intended.
Tests were written for different combinations of functions which were used together in the system using a variety of different inputs, and the output of these tests was compared with the expected output.
As with the unit tests discussed in section \ref{unit_testing}, the \textit{Python} library \textit{unittest} was used to write these tests.
The results of the integration tests can be seen in a simplifed format in table \ref{integration_test_results}.

\begin{table}[t]
  \caption{Results of Integration Tests}
  \begin{center}
    \begin{tabularx}{\textwidth}{|Y|c|} \hline
      % \textbf{Integration Test} & \makecell{\textbf{Linked \\ Requirement(s)}} & \textbf{Success / Failure} \\ \thickhline
      \textbf{Integration Test} & \textbf{Success / Failure} \\ \thickhline
      Test \textit{TreeTopoGeneric} class with \textit{get\_poisson\_delay} function class & \textcolor{green}{SUCCESS} \\ \hline
      Test \textit{TreeTopoGeneric} class with \textit{get\_poisson\_delay} function class & \textcolor{green}{SUCCESS} \\ \hline
      Test \textit{TreeTopoGeneric} class with \textit{halve\_delay} function & \textcolor{green}{SUCCESS} \\ \hline
      Test \textit{TreeTopoGeneric} class with \textit{test\_cloud\_fpga} function & \textcolor{green}{SUCCESS} \\ \hline
      Test \textit{TreeTopoGeneric} class with \textit{get\_poisson\_delay} function class and \textit{halve\_delay} functions & \textcolor{green}{SUCCESS} \\ \hline
      Test \textit{TreeTopoGeneric} class with \textit{test\_cloud\_fpga} and \textit{halve\_delay} functions & \textcolor{green}{SUCCESS} \\ \hline
      Test \textit{TreeTopoGeneric} class with \textit{test\_cloud\_fpga}, \textit{get\_poisson\_delay} and \textit{halve\_delay} functions & \textcolor{green}{SUCCESS} \\ \hline


      % 0 & 0 & 0 & 0 (A) \\ \hline
    \end{tabularx}
  \end{center}
  \label{integration_test_results}
\end{table}

\section{System Testing}
\label{system_testing}
System testing was conducted to ensure the system as a whole performed as intended.
The entire system was tested using different inputs for the available configuration options, and the output of the system was compared with the expected output.
The results of the system tests can be seen in table \ref{integration_test_results}.


\begin{table}[t]
  \caption{Results of System Tests}
  \begin{center}
    \begin{tabularx}{\textwidth}{|Y|c|c|} \hline
      % \textbf{System Test} & \makecell{\textbf{Linked \\ Requirement(s)}} & \textbf{Success / Failure} \\ \thickhline
      \textbf{System Test} & \textbf{Linked Requirement(s)} & \textbf{Success / Failure} \\ \thickhline
        Can the system function using a tree topology with a spread of two and a depth of four, with a total of 8 hosts? & F1, F2 & \textcolor{green}{SUCCESS} \\ \hline
        Can the system function using a tree topology with a spread of three and a depth of 8, with a total of 2187 hosts? & F1, F2, F3, F4 & \textcolor{green}{SUCCESS} \\ \hline
        Can the system report the bandwidth in the network? & F7 & \textcolor{green}{SUCCESS} \\ \hline
        Can the bandwidth in the network be configured correctly? & F9 & \textcolor{green}{SUCCESS} \\ \hline
        Can the link delay in the network be configured correctly? & F10 & \textcolor{green}{SUCCESS} \\ \hline
        Can the probability of packet loss in the network be configured correctly? & F11 & \textcolor{green}{SUCCESS} \\ \hline
        Can the link delay in the network be configured using a Poisson distribution correctly? & F16 & \textcolor{green}{SUCCESS} \\ \hline
        Can all node connections be correctly displayed? & F20 & \textcolor{green}{SUCCESS} \\ \hline
        Can the user specify the level of logging employed by the system? & F13 & \textcolor{green}{SUCCESS} \\ \hline

      % 0 & 0 & 0 & 0 (A) \\ \hline
    \end{tabularx}
  \end{center}
  \label{system_test_results}
\end{table}


\section{Environment Testing}
\label{environment_testing}
Environment testing was conducted to ensure the system performed as expected in different environments.
The system was tested using a variety of different operating systems with both \textit{Python} 2.7 and \textit{Python} 3.6, and the output of the system was compared with the expected output.
The results of the system tests can be seen in table \ref{integration_test_results}.

\begin{table}[t]
  \caption{Results of Environment Tests}
  \begin{center}
    \begin{tabularx}{\textwidth}{|Y|Y|Y|} \hline
      % \textbf{Environment Test} & \makecell{\textbf{Linked \\ Requirement(s)}} & \textbf{Success / Failure} \\ \thickhline
      \textbf{Python Version} & \textbf{OS} & \textbf{Success / Failure} \\ \thickhline
      2.7 & Ubuntu 16.04 LTS & \textcolor{green}{SUCCESS} \\ \hline
      3.6 & Ubuntu 16.04 LTS & \textcolor{green}{SUCCESS} \\ \hline
      2.7 & Ubuntu 14.04 LTS & \textcolor{green}{SUCCESS} \\ \hline
      3.6 & Ubuntu 14.04 LTS & \textcolor{green}{SUCCESS} \\ \hline
      % 0 & 0 & 0 & 0 (A) \\ \hline
    \end{tabularx}
  \end{center}
  \label{environment_test_results}
\end{table}

\section{User Testing}
User testing was conducted to ensure that the system was sufficiently easy to use, and to find any issues with the system not identified through other means of testing.
The system was presented to a number of individuals who were asked to use the system without further instruction, and were then observed.
Feedback received through this method is shown in table \ref{user_test_results}, including the steps taken to resolve any issues identified.

\begin{table}[t]
  \caption{Results of User Tests}
  \begin{center}
    \begin{tabularx}{\textwidth}{|Y"Y|} \hline
      \textbf{User Feedback} & \textbf{Developer Response} \\ \thickhline
      It is not clear how to install the system. & An installation guide was added to the public \textit{GitHub} page for the system. \\ \hline
      It is not clear what the available customisation options are.  & While a guide to the available customisation options can be found by passing the \textit{--help} flag to the system, an equivalent guide was added to the public \textit{GitHub} page for the system. \\ \hline
      % 0 & 0 & 0 & 0 (A) \\ \hline
    \end{tabularx}
  \end{center}
  \label{user_test_results}
\end{table}
