% Python is slow
% People have slow computers
% Mininet only runs on ubuntu 16
% Python 2
% Switches 1U rack (usually)
% switches standard rack depth (but normally half or less)
% Python minimum operating reqs
% mininet minimum operating reqs
% Deadlines etc
% Python 2 deprecated
% Ubuntu 14.04 EOL
% Ubuntu 16.04 EOL

This project is subject to a number of constraints which limit the capabilities or scope of the project.
These have been separated into the constraints which affected the implemented aspect of the project and the constraints which would apply to an FPGA-based smart network switch.
% These have been separated into the constraints affected the implemented aspect of the project, the constraints which would apply to an FPGA-based smart network switch, and the constraints on the management aspect of the project.

\subsection{System Constraints}
The constraints described in this section affected the implemented aspect of the system. As a dependency of the system, \textit{Mininet} \cite{mininet} imposed a number of constraints, largely relating to the software of the system. \textit{Mininet}'s API is written in \textit{Python}, which restricts the system to using \textit{Python} as well. Naturally it would be possible to combine an alternative language with Python, and simply use \textit{Python} to interact with the API, however this would introduce an additional level of complexity to the project with little potential benefit, since any improvements in speed offered by a different language would be bottlenecked by the \textit{Python} API.

The most recent stable version of \textit{Mininet}, 2.2.2, uses \textit{Python} 2.7, an older version of Python due to be deprecated on 1 January 2020. While the next stable release of \textit{Mininet} is expected to be updated to use \textit{Python} 3, this has required the project to also use \textit{Python} 2.7.

\textit{Mininet} 2.2.2 contains a known issue with Ubuntu 16.04 or newer due to an incompatibility with the Linux kernel versions 3.16 and higher \cite{mininet_2_2_2_release_notes}. As a result, \textit{Mininet} recommends using Ubuntu 14.04 LTS, which was released in April 2014. Normal LTS support for Ubuntu 14.04 ended on 25 April 2019, at which point futher security updates for the OS became only available to those who have purchased ``Extended Security Maintenance'' for it \cite{ubuntu_14_04_release}. As mentioned in section \ref{mininet}, the next stable release of \textit{Mininet} is expected to be supported on Ubuntu 18.04 LTS.

The system is intended to be lightweight and available to all users, so it needs to be capable of running on systems with a relatively low specification, and should not take a long time to run.

\subsection{FPGA-based Smart Network Switch Constraints}
The constraints described in this section would affect an FPGA-based smart network switch. One constraint on these devices is the physical size of the hardware. In order to conform to industry standards, an FPGA-based smart network switch would need to be designed such that it could be mounted in a standard 19 inch server rack.
The height of these server racks and the devices mounted within them are measured in `rack units', or `U's, with a standard full size rack being 42U tall.
Network switches are rarely largely than 1U, and as a result an FPGA-based smart network switch should also be 1U, in order to make it easier for users to transition from a standard network switch to an FPGA-based smart network switch.

As discussed in section \ref{physical_media_research}, modern networks use two primary mediums for wired data transfer, UTP copper cables, or fibre optic cables. An FPGA-based smart network switch will be constrained to use one (or both) of these mediums.


% \subsection{Management Constraints}
% These are the constraints for me doing the project
