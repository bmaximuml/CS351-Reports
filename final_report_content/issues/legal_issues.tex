% Use of third party tools
% Licensing
% aim to use open source
% Data Protection Act 1998
% GDPR
%
% how was data obtained?
% "I abided by terms of use"
% If you agreed to any terms and conditions, what do they say? are there consequences?
%
% liability:
%   could a user injure themselves while useing system (obvs not)
%   could a user damage their network using system?
% Copyright?
% patenting
%
% APIs
% watermarking
% what license are you providing the software to be used?
% software license agreement?

When using third tools, it is important to ensure that they are being use in according with the license they are provided.
Much open-source software is provided with either the \textit{Apache License 2.0} \cite{apache_2_0}, the \textit{GNU General Public License v3.0} \cite{gnu_gpl_3}, or other similar licenses allowing conditional free use of the software.
As a result, this project will aim to use open-source tools where they are available to avoid issues with copyright.

Two acts of legislation this project will be additionally careful not to contravene are the \textit{General Data Protection Regulation 2016} \cite{eu_2016_679} and the \textit{Data Protection Act 2018} \cite{uk_dpa_2018}. These acts of legislation involve the storage, collection, and processing of data, and relate back to the issues descibed in sections \ref{ethical_issues} and \ref{social_issues}.

Wherever secondary data has been used, i.e. data not directly collected by the developer, any conditions to the use of this data must be considered. Should any terms of use relating to the data be signed, these must be kept to, and the primary stakeholders outlined in section \ref{primary_stakeholders} should be aware of any consequences that could occur from not keeping to these terms of use.
