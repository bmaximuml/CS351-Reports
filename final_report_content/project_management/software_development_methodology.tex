% very very agile
% due to flexible objectives
% Regular meetings with supervisor (like scrums)
  % helped with direction of project

% why not waterfall?
% changing requirements, changing title, tools not available


This project uses an agile methodology in order to adapt to changes arising during the project. This approach proved invaluable since the direction of the project went through a series of significant changes since it started. Weekly meetings between the developer and the project supervisor over the course of the project allowed for frequent, short term targets to be set, and allowed for the project to be regularly reevaluated to ensure that it was still viable in its current state, and to make adjustments otherwise.

A less flexible methodology, such as a waterfall methodology, would have likely resulted in the project's failure, due to the significant changes which occured after the initial specification was made.


% From spec
% This project will use an Agile methodology so that it can adapt to changes which arise during the project. Since the research and implementation stages of the project will contribute to confirming its direction, this flexibilty is important. In addition, git \cite{git} will be used to track changes in both written documents and code developed, such as P4 or Verilog code. Repositories will be set up using git for the different areas of the project, and these repositories will be stored primarily on an online GitHub \cite{github} server that will be backed up regularly.
% The Gantt chart in Figure \ref{gantt_chart} has been constructed to show an outline of the project timetable, and is intentionally flexible for the changes that will take place.
