% give a brief intro to risk
  % From coles: "With any large project that spans over a long period of time there are many risk that, should they occur, could be catastrophic for the results of the project."

% Lack of access to tools
% Those guys who we wanted to get the P4 resources never delivered

% Can you make a risk thing from project management module?
  % With severity and probability

% Maybe a table?
% Separate into risk : mitigation strategy

% Complete failure of main device (i.e. Iroh) :  Backups were taken regularly, and alternative devices were available
% github dies : local copy availble, resilio sync backups avilable, will switch to gitlab.com
% mininet stops being supported : existing version of mininet still available, development limited to Ubuntu 16.04 (are there alternatives?)

% Changes to any third party libraries? : find alternatives or stick with most recent versions
% Single developer : agile, and regular supervisor meetings. constantly ensuring goals were feasible in case developer got ill or somethign
% ESLP issues : BCS guidelines, and anonymity
% General hardware failure : No individual piece of hardware crucial

In order to ensure the success of the project, a number of potentially negative situations have been considered. In order to minimise the impact these situations could have on the project, strategies have been established to mitigate them.

These situations are shown in table \ref{risk_mitigation_table}, along with a measure of the severity of the situation and a measure of the likelihood of the situation.

\begin{table}[t]
  \begin{adjustbox}{addcode={\begin{minipage}{\width}}{\caption{
      Possible Risks and Mitigation Strategies
      }\label{risk_mitigation_table}\end{minipage}},rotate=90,center}
      % }\label{final_gantt_chart}\end{minipage}},rotate=90,scale={0.7}{0.8},center}
  % \caption{Possible Risks and Mitigation Strategies}
  % \begin{center}
    \begin{tabularx}{\textheight}{|l|X|l|l|X|} \hline
      \textbf{Risk} & \textbf{Description} & \textbf{Severity} & \textbf{Likelihood} & \textbf{Mitigation} \\ \thickhline
      \makecell[l]{Hardware \\ failure} &
      Complete failure of primary machine used for project development.
      This would then mean an alternative device would need to be procured to use as the primary development machine, and any files existing only on this machine would be lost. &
      Medium &
      Low &
      All files were backed up using both \textit{GitHub} and \textit{Resilio Sync}.
      As a result, no files would be lost if the machine failed.
      While finding an alternative machine for development would be inconvenient, other machines are available, such as those in the Warwick Department of Computer Science. \\ \hline

      \makecell[l]{\textit{GitHub} \\ failure} &
      Failure of the remote repository where project code and documents are stored. This would cause the loss of any files in the \textit{GitHub} repository. &
      Low &
      Low &
      Since any files in the \textit{GitHub} repository will have been uploaded to it from a local machine, at no point should there be any data in \textit{GitHub} which does not exist locally.
      In addition, \textit{Resilio Sync} will be used to replicate files across multiple local devices, so that backups exist of all code and documents.
      An alternative web-based hosting service for version control using \textit{Git}, such as \textit{GitLab} \cite{gitlab}, will be used in the event that \textit{GitHub} does fail. \\ \hline

      \makecell[l]{\textit{Mininet} \\ discontinued} &
      \textit{Mininet} is a significant dependency of the implemented system.
      If development of \textit{Mininet} stopped, a suitable alternative may need to be found.
      While other network simulation software does exist, finding a replacement for \textit{Mininet} which is open-source, and has a suitable API may be challenging.
      Fortunately, the implemented system would still be able to use the latest version of \textit{Mininet}, so it may be possible to continue development with this version. &
      Medium &
      Low &
      A copy has been made of \textit{Mininet}'s \textit{GitHub} repository so that it can still be used in the event that \textit{Mininet} is discontinued. \\ \hline

      % Single developer &
      % The project was being developed by an individual, resulting in a greater impact to the project if the developer became temporarily unable to work, for example due to illness. &
      % Medium &
      % Medium &
      % The agile methodology chosen for the project provides the project with the necessary flexibility to deal with this situation. Frequent meetings with the project supervisor to re-evaluate the feasiblilty of the project, taking into account any illness or similar circumstances, should enable the project to complete on time. \\ \hline

    \end{tabularx}
  \end{adjustbox}
  % \end{center}
  % \label{risk_mitigation_table}
\end{table}


\begin{table}[t]
  \begin{adjustbox}{rotate=90,center}
    \begin{tabularx}{\textheight}{|l|X|l|l|X|} \hline
      \textbf{Risk} & \textbf{Description} & \textbf{Severity} & \textbf{Likelihood} & \textbf{Mitigation} \\ \thickhline
      \makecell[l]{Single \\ developer} &
      The project was being developed by an individual, resulting in a greater impact to the project if the developer became temporarily unable to work, for example due to illness. &
      Medium &
      Medium &
      The agile methodology chosen for the project provides the project with the necessary flexibility to deal with this situation. Frequent meetings with the project supervisor to re-evaluate the feasiblilty of the project, taking into account any illness or similar circumstances, should enable the project to complete on time. \\ \hline

    \end{tabularx}
  \end{adjustbox}
  % \end{center}
  % \label{risk_mitigation_table}
\end{table}

% \begin{center}
% \begin{table}[t]
%   \begin{adjustbox}{addcode={\begin{minipage}{\width}}{\caption{
%     Possible Risks and Mitigation Strategies
%   %       Possible Risks and Mitigation Strategies
%         }\label{risk_mitigation_table}\end{minipage}},rotate=90,center}
%   \begin{longtable}{|l|p{3cm}|l|l|p{3cm}|} \\ \hline
%       \textbf{Risk} & \textbf{Description} & \textbf{Severity} & \textbf{Likelihood} & \textbf{Mitigation} \\ \thickhline
%       \makecell[l]{Hardware \\ failure} &
%       Complete failure of primary machine used for project development.
%       This would then mean an alternative device would need to be procured to use as the primary development machine, and any files existing only on this machine would be lost. &
%       Medium &
%       Low &
%       All files were backed up using both \textit{GitHub} and \textit{Resilio Sync}.
%       As a result, no files would be lost if the machine failed.
%       While finding an alternative machine for development would be inconvenient, other machines are available, such as those in the Warwick Department of Computer Science. \\ \hline
%
%       \makecell[l]{\textit{GitHub} \\ failure} &
%       Failure of the remote repository where project code and documents are stored. This would cause the loss of any files in the \textit{GitHub} repository. &
%       Low &
%       Low &
%       Since any files in the \textit{GitHub} repository will have been uploaded to it from a local machine, at no point should there be any data in \textit{GitHub} which does not exist locally.
%       In addition, \textit{Resilio Sync} will be used to replicate files across multiple local devices, so that backups exist of all code and documents.
%       An alternative web-based hosting service for version control using \textit{Git}, such as \textit{GitLab} \cite{gitlab}, will be used in the event that \textit{GitHub} does fail. \\ \hline
%
%       \makecell[l]{\textit{Mininet} \\ discontinued} &
%       \textit{Mininet} is a significant dependancy of the implemented system.
%       If development of \textit{Mininet} stopped, a suitable alternative may need to be found.
%       While other network simulation software does exist, finding a replacement for \textit{Mininet} which is open-source, and has a suitable API may be challenging.
%       Fortunately, the implemented system would still be able to use the latest version of \textit{Mininet}, so it may be possible to continue development with this version. &
%       Medium &
%       Low &
%       A copy has been made of \textit{Mininet}'s \textit{GitHub} repository so that it can still be used in the event that \textit{Mininet} is discontinued. \\ \hline
%
%       Single developer &
%       The project was being developed by an individual, resulting in a greater impact to the project if the developer became temporarily unable to work, for example due to illness. &
%       Medium &
%       Medium &
%       The agile methodology chosen for the project provides the project with the necessary flexibility to deal with this situation. Frequent meetings with the project supervisor to re-evaluate the feasiblilty of the project, taking into account any illness or similar circumstances, should enable the project to complete on time. \\ \hline
%   \end{longtable}
% \end{adjustbox}
% % \end{center}
% \end{table}
