\documentclass[12pt, a4paper, twoside]{IEEEtran}

%% Language and font encodings
\usepackage[english]{babel}
\usepackage[utf8x]{inputenc}
\usepackage[T1]{fontenc}
\usepackage{mathptmx}

%% Sets page size and margins
\usepackage[a4paper,top=2cm,bottom=1.6cm,left=1.8cm,right=1.8cm,marginparwidth=2cm]{geometry}

%% Useful packages
\usepackage{amsmath}
\usepackage{graphicx}
\usepackage{pgfgantt}
\usepackage{url}
\usepackage{xparse}
\usepackage{listings}
\usepackage[colorinlistoftodos]{todonotes}
\usepackage[colorlinks=true, allcolors=blue]{hyperref}

%% Section Numbering
%\setcounter{secnumdepth}{0} % no sections will be numbered
%\setcounter{secnumdepth}{1} % only sections will be numbered
%\setcounter{secnumdepth}{2} % sections and subsections will be numbered
%\setcounter{secnumdepth}{3} % sections, subsections and subsubsections will be numbered

%% TrueType font for code. use \codeword{} around sections of code in your document
\NewDocumentCommand{\codeword}{v}{
  \texttt{\textcolor{black}{#1}}
}

% You may want to use \vspace{-1cm} at times to change vertical spacing in a slightly hacky but functional way

\title{CS351 Computer Systems Engineering Project \\ Network Switch Design on FPGAs \\ \Large{Specification}}
\author{1611586}

\begin{document}
\maketitle

\section{Problem Statement}
\label{problem_statement}
Networking is an important area of Computing, and as network sizes and average complexity of projects have increased,
the need for network connections with very low latency has also increased.
Using FPGAs to conduct switching logic can result in a switch with a very high throughput, since this switching logic will be implemented in hardware. The look-up tables (LUTs) on an FPGA can be used to store MAC addresses, allowing an FPGA based switch to theoretically determine the appropriate port to direct a packet to in a single clock cycle. Unlike ASICs, FPGAs are also able to be reconfigured on the fly, so MAC address tables can dynamically updated while still being implemented in hardware.
For all of these reasons, FPGA-based network switches should be able to be used to create networks with a much higher throughput than traditional networking hardware, and investigation into FPGA-based switches may reveal further opportunity for other FPGA-based networking hardware, such as routers or firewalls.

\section{Objectives}
\label{objectives}
\begin{itemize}
  \item Research the OSI network model
  \item Research the NetFPGA platform \cite{NetFPGA}
  \item Research the packet switching language P4 \cite{P4}
  \item Implement a packet analyser on a NetFPGA
  \item Implement a packet switcher on a NetFPGA
  \item Test throughput and latency of switching packets using the NetFPGA packet switcher
  \item Test throughput and latency of switching packets using a conventional network switch
  \item Compare performance of NetFPGA packet switcher to conventional network switch
  \item Write up performance comparison
\end{itemize}

\section{Methods}
\label{methods}
This project will use an agile methodology so that it can adapt to changes which arise during the project. Since the research and implementation stages of the project will contribute to confirming the direction the project will take, this flexibilty is important. In addition, git \cite{git}

\section{Timetable}
\label{timetable}
\begin{ganttchart}{1}{31}
  \gantttitle{Term 1}{10}
  \gantttitle{Christmas}{4}
  \gantttitle{Term 2}{10}
  \gantttitle{Easter}{5}
  \gantttitle{Term 3}{2} \\
  \gantttitlelist{1,...,10}{1}
  \gantttitlelist{1,...,4}{1}
  \gantttitlelist{1,...,10}{1}
  \gantttitlelist{1,...,5}{1}
  \gantttitlelist{1,...,2}{1} \\
  \ganttgroup{Group 1}{1}{7} \\
  \ganttbar{Task 1}{1}{2} \\
  \ganttlinkedbar{Task 2}{3}{7} \ganttnewline
  \ganttmilestone{Milestone}{7} \ganttnewline
  \ganttbar{Final Task}{8}{12}
  \ganttlink{elem2}{elem3}
  \ganttlink{elem3}{elem4}
\end{ganttchart}

\section{Resources}
\label{resources}

\section{Glossary}
\label{glossary}
\begin{itemize}
  \item \textbf{FPGA}: Field-Programmable Gate Array
  \item \textbf{ASIC}: Application-Specific Integrated Circuit
  \item \textbf{LUT}: Look-Up Table
\end{itemize}


\bibliographystyle{ieeetr}
\bibliography{bibliography}

\end{document}
